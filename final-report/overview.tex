\section{Design Overview}
\label{sec:overview}
Our architecture has two main components: the virtualization framework, and the integration front-end.  The virtualization framework contains a container-ready kernel as well as each of the spawned containers (see Figure \ref{fig:arch}).  A container could contain one or more applications.  Our initial design goals were to explore virtualization of x86-based operating systems as well as ARM based systems.  However, our results \ref{sec:discuss} show that these solutions will be far too slow and hence we choose to use OS-level virtualization techniques.\\

The other component is the integration front-end. This contains a light X-server that has been integrated into the host OS, and also contains the container controller. The controller runs inside the host OS. The controller will be the user interface that controls launching, switching, suspending, resuming, migrating the containers, as well as ensuring the X-server is running properly. \\

A shared X-server removes rendering overhead from each container and reduces the complexity in composing the UI. We choose to share the rendering state between third-party applications for performance reasons and simplicity of design, but ensure isolation from native applications. \\
%Additionally the X-server runs native code, which allows for it to be less CPU intensive than running inside a container. \\

In \emph{\proj}, the virtualization framework is designed to be device independent. The integration front-end will have to be ported for each new device we support, but its implementation is relatively simple. Our first implementation focuses on one particular device (Palm Pre) and OS (WebOS). \\

In summary, we leverage the existing linux containers architecture to build a secure, usable and portable framework for mobile device virtualization. Our contributions are focused on providing the ability for secure execution of applications while also providing integration to the mobile device. Our proposed security model is an effective method to isolate applications and we find that our design ideas are similar to other recent efforts \cite{grier2008secure} in isolating untrusted code. 
