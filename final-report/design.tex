\section{Design}
\label{sec:design}

\subsection{Virtualization on ARM}
The two primary goals of our design is to a. Provide isolation of selected applications on mobile phones and b. To build a usable solution which can be integrated seamlessly into the host operating system of the smartphone. To achieve the first goal, we first investigated Type-II virtualization techniques which allow a guest operating system to run from within the host operating system. The ARM architecture is not directly virtualizable as there privileged instructions which when executed by the guest operating system, do not trap to the host kernel. There are many existing techniques like dynamic binary translation, trap-and-emulate using hardware extensions, translate to trap, paravirtualization which have been used in existing virtualization solutions for x86.


\subsection{Linux Containers}
Linux Containers (LXC) implement OS-level virtualization techniques in order to run a number of isolated virtual environments on a single host.  LXC differs from conventional virtualization techniques, which generally require the installation of guest OSes.  The isolated virtual environments or containers, are built upon other Linux security mechanisms. We list the different resources isolated using containers and the techniques used for them.

\subsubsection{Host Identification}
On Linux machines, the identity of a system is usually established through the command `uname'. The command reads informations stored in the structure `utsname'. This includes the name of the operating system in use, the current release and version of the operating system. The structure also contains the hostname of the machine and is used to identify the machine for all network communications.

\subsection{Process Identifiers}


\subsection{Inter-Process Communications}


\subsection{User Namespaces}


\subsection{Network Devices}


\subsection{Readonly-bind mounts}


\subsubsection{Copy-on-Write filesystems}


Resource namespaces allow containers to manipulate the accessing of processes, files, and hardware resources.  Control groups are used to limit the resources used by a container.  Capability bounding sets reduce a container's privileges.  

Unlike with conventional virtualization, LXC has almost no overhead as it does not require guest OSes or dynamic binary translation.  On the other hand, because it is a OS-level virtualization technique and runs processes on a Linux kernel, other OSes such as Windows and other UNIXes (BSD or OSX) can not be used as containers.  

The main obstacle to using Linux Containers are the requirements.  LXC requires a Linux kernel version 2.6.27 or greater, and the Pre is running 2.6.24.  The kernel functionality would have to be back ported in order to work with the Pre.  The Android is running 2.6.29; however, porting the LXC tools to Android will likely be very difficult.
